% Options for packages loaded elsewhere
\PassOptionsToPackage{unicode}{hyperref}
\PassOptionsToPackage{hyphens}{url}
%
\documentclass[
]{article}
\usepackage{amsmath,amssymb}
\usepackage{lmodern}
\usepackage{iftex}
\ifPDFTeX
  \usepackage[T1]{fontenc}
  \usepackage[utf8]{inputenc}
  \usepackage{textcomp} % provide euro and other symbols
\else % if luatex or xetex
  \usepackage{unicode-math}
  \defaultfontfeatures{Scale=MatchLowercase}
  \defaultfontfeatures[\rmfamily]{Ligatures=TeX,Scale=1}
\fi
% Use upquote if available, for straight quotes in verbatim environments
\IfFileExists{upquote.sty}{\usepackage{upquote}}{}
\IfFileExists{microtype.sty}{% use microtype if available
  \usepackage[]{microtype}
  \UseMicrotypeSet[protrusion]{basicmath} % disable protrusion for tt fonts
}{}
\makeatletter
\@ifundefined{KOMAClassName}{% if non-KOMA class
  \IfFileExists{parskip.sty}{%
    \usepackage{parskip}
  }{% else
    \setlength{\parindent}{0pt}
    \setlength{\parskip}{6pt plus 2pt minus 1pt}}
}{% if KOMA class
  \KOMAoptions{parskip=half}}
\makeatother
\usepackage{xcolor}
\usepackage{longtable,booktabs,array}
\usepackage{calc} % for calculating minipage widths
% Correct order of tables after \paragraph or \subparagraph
\usepackage{etoolbox}
\makeatletter
\patchcmd\longtable{\par}{\if@noskipsec\mbox{}\fi\par}{}{}
\makeatother
% Allow footnotes in longtable head/foot
\IfFileExists{footnotehyper.sty}{\usepackage{footnotehyper}}{\usepackage{footnote}}
\makesavenoteenv{longtable}
\setlength{\emergencystretch}{3em} % prevent overfull lines
\providecommand{\tightlist}{%
  \setlength{\itemsep}{0pt}\setlength{\parskip}{0pt}}
\setcounter{secnumdepth}{-\maxdimen} % remove section numbering
\ifLuaTeX
  \usepackage{selnolig}  % disable illegal ligatures
\fi
\IfFileExists{bookmark.sty}{\usepackage{bookmark}}{\usepackage{hyperref}}
\IfFileExists{xurl.sty}{\usepackage{xurl}}{} % add URL line breaks if available
\urlstyle{same} % disable monospaced font for URLs
\hypersetup{
  hidelinks,
  pdfcreator={LaTeX via pandoc}}

\author{}
\date{}

\begin{document}

\hypertarget{fiche-de-traitement-du-signal}{%
\section{fiche de traitement du
signal}\label{fiche-de-traitement-du-signal}}

\hypertarget{transformuxe9e-de-fourrier}{%
\subsection{transformée de Fourrier}\label{transformuxe9e-de-fourrier}}

On prend \(p \in \mathcal{L}^2\) (ou \(p \in \mathcal{L}^1\),
c\textquotesingle est \emph{plus simple}) Formule directe \[
\mathcal{F}[p](f) := \int_\mathbb{R} e^{-j2\pi f t} p(t) \mathrm{d}t
\] {[}{[}transformée de Fourrier inverse\textbar Formule inverse{]}{]}
\[
\mathcal{F^{-1}}[P](t) := \int_\mathbb{R} e^{j2\pi f t} P(f) \mathrm{d}f
\]

\href{https://youtube.com/watch?v=spUNpyF58BY}{3b1b}

\hypertarget{duxe9rivuxe9eduxe9rivation}{%
\subsubsection{{[}{[}dérivée\textbar Dérivation{]}{]}}\label{duxe9rivuxe9eduxe9rivation}}

\[
\mathcal{F}[x^{(n)}](f) = (j2\pi f)^n \mathcal{F}[x](f)
\]

\hypertarget{propriuxe9tuxe9s}{%
\subsubsection{Propriétés}\label{propriuxe9tuxe9s}}

\(\mathcal{F}\) est

\begin{itemize}
\tightlist
\item
  linéaire
\item
  conserve la parité
\item
  \textbf{transformation}
  \(\mathcal{F}[x(t-t_0)](f) = e^{-2\pi j f t_0} X(f)\)
\item
  \textbf{modulation} \(\mathcal{F}[x(t)e^{j2\pi f_0 t}] = X(f-f_0)\)
\item
  \textbf{similitude}
  \(\mathcal{F}[x(at)](f) = \frac{1}{|a|} X(\frac{f}{a})\)
\end{itemize}

en gros translation en temps \(\iff\) multiplication par exp(...) en
fréquence

\hypertarget{transformuxe9es-usuelles}{%
\subsubsection{Transformées usuelles}\label{transformuxe9es-usuelles}}

\begin{longtable}[]{@{}ll@{}}
\toprule()
Temporel & Fréquentiel \\
\midrule()
\endhead
\(x^\ast\) & \(X^\ast \circ (-\operatorname{id})\) \\
\(x \circ (a \operatorname{id} + b)\) & \$\textbackslash frac\{1\}\{ \\
\bottomrule()
\end{longtable}

\hypertarget{transformuxe9e-de-fourrier-inverse}{%
\subsection{transformée de Fourrier
inverse}\label{transformuxe9e-de-fourrier-inverse}}

\#signal

Avec \(X\) dans \(L^1\) ou \(L^2\)

\[
\int_\mathbb{R} e^{j2\pi ft} X(f) \mathrm{d}f
\]

\hypertarget{produit-de-convolution}{%
\subsection{produit de convolution}\label{produit-de-convolution}}

\#signal Opérateur défini par

\[
f \ast g = \int_{-\infty}^{+\infty} g(t-\tau) f(\tau) \mathrm{d}\tau
\]

\hypertarget{propriuxe9tuxe9s-1}{%
\subsubsection{Propriétés}\label{propriuxe9tuxe9s-1}}

\begin{itemize}
\item
  Avec la {[}{[}transformée de Fourrier{]}{]}:

  \begin{itemize}
  \tightlist
  \item
    \(\mathcal{F}[f \ast g] = \mathcal{F}[f]\cdot \mathcal{F}[g]\)
  \item
    \(\mathcal{F}[f\cdot g] = \mathcal{F}[f] \ast \mathcal{F}[g]\)
  \end{itemize}
\item
  \(\mathcal{F}[\text{carré}] = \text{sinc}\)
\item
  {[}{[}égalité de Parseval{]}{]}
\end{itemize}

\hypertarget{avec-limpulsion-de-dirac}{%
\subsubsection{Avec l\textquotesingle{[}{[}impulsion de
Dirac{]}{]}}\label{avec-limpulsion-de-dirac}}

\begin{itemize}
\tightlist
\item
  \textbf{localisation} \(x(t)\delta(t-t_0) = x(t_0)\delta(t-t_0)\)
\item
  produit de convolution \(x(t)\ast \delta(t-t_0) = x(t-t_0)\)
\item
  \(x(t_0) = \int_\mathbb{R} x(t) \delta(t-t_0) \mathrm{d}t\)
\end{itemize}

\hypertarget{convolution-par-la-ruxe9ponse-impulsionnelle-dun-filtre}{%
\subsubsection{Convolution par la {[}{[}réponse impulsionnelle{]}{]}
d\textquotesingle un
{[}{[}filtre{]}{]}}\label{convolution-par-la-ruxe9ponse-impulsionnelle-dun-filtre}}

\hypertarget{propriuxe9tuxe9s-2}{%
\paragraph{Propriétés}\label{propriuxe9tuxe9s-2}}

Soit \(h\) la réponse impulsionnelle d\textquotesingle un filtre.
L\textquotesingle opération \(T = x \mapsto x \ast h\) est:

\begin{itemize}
\tightlist
\item
  Linéaire
\item
  Invariante dans le temps
  \[\forall t_0, t \in \mathbb{R}, T[x](t) = T[x](t-t_0)\]
\item
  Stable (BIBO) Si \(x\) est borné alors \(T[x]\) est borné, ou de
  manière équivalente (CNS) \[\int_\mathbb{R} |h| < \infty\]
\item
  Le spectre du signal est "limité"
\end{itemize}

\hypertarget{impulsion-de-dirac}{%
\subsection{impulsion de Dirac}\label{impulsion-de-dirac}}

\#signal

\(\delta\) tel que

\begin{itemize}
\tightlist
\item
  \(\int_{-\infty}^0 \delta = 0\)
\item
  \(\int_{0}^{+\infty} \delta = 1\)
\end{itemize}

\hypertarget{transformuxe9e-de-fourrier-1}{%
\subsubsection{{[}{[}transformée de
Fourrier{]}{]}}\label{transformuxe9e-de-fourrier-1}}

\begin{itemize}
\tightlist
\item
  \(\mathcal{F}(\delta) = 1\)
\item
  \(\mathcal{F}(1) = \delta\)
\item
  \(\mathcal{F}(\delta(t-t_0)) = \operatorname{exp}(-j2 \pi ft_0)\) (et
  pareil dans l\textquotesingle autre sens)
\end{itemize}

\hypertarget{propriuxe9tuxe9s-3}{%
\subsubsection{Propriétés}\label{propriuxe9tuxe9s-3}}

\begin{itemize}
\tightlist
\item
  {[}{[}série de Fourier{]}{]}: \[
  \sum_{n\in \mathbb{Z}} c_n e^{+j2\pi n f_0 t} = \sum_{n\in \mathbb{Z}} c_n \delta(f-nf_0)
  \]
\end{itemize}

\hypertarget{signal-duxe9terministe-uxe0-uxe9nergie-finie}{%
\subsection{signal déterministe à énergie
finie}\label{signal-duxe9terministe-uxe0-uxe9nergie-finie}}

\#signal

\hypertarget{duxe9finition}{%
\subsubsection{Définition}\label{duxe9finition}}

Un {[}{[}signal{]}{]} {[}{[}signal
déterministe\textbar déterministe{]}{]} à {[}{[}énergie
d\textquotesingle un signal\textbar énergie{]}{]}
{[}{[}fini\textbar finie{]}{]}

\hypertarget{uxe9nergie}{%
\subsubsection{Énergie}\label{uxe9nergie}}

\[
E(x) = \int_\mathbb{R} | x |^2 = \int_\mathbb{R} |\mathcal{F}[X]|^2 
\]

\hypertarget{produit-scalaire}{%
\subsubsection{Produit scalaire}\label{produit-scalaire}}

\[
\langle x, y \rangle = \int_\mathbb{R} x y^\ast
\]

\hypertarget{densituxe9-spectrale-duxe9nergie}{%
\subsubsection{{[}{[}densité spectrale
d\textquotesingle énergie{]}{]}}\label{densituxe9-spectrale-duxe9nergie}}

\[
s_x(f) = |\mathcal{F}[x](f)|^2
\]

\hypertarget{signal-duxe9terministe-puxe9riodique-uxe0-puissance-finie}{%
\subsection{signal déterministe périodique à puissance
finie}\label{signal-duxe9terministe-puxe9riodique-uxe0-puissance-finie}}

\hypertarget{duxe9finition-1}{%
\subsubsection{Définition}\label{duxe9finition-1}}

Un {[}{[}signal{]}{]} {[}{[}signal déterministe{]}{]}
{[}{[}périodique{]}{]} à {[}{[}énergie d\textquotesingle un
signal\textbar puissance{]}{]} {[}{[}fini\textbar finie{]}{]}.

\hypertarget{puissance}{%
\subsubsection{Puissance}\label{puissance}}

Soit \(x\) un signal déterministe périodique à puissance finie de
{[}{[}période{]}{]} \(T\)

\[
P(x) = \frac{1}{T} \int^{-T/2}_{T/2} |x|^2
\]

\hypertarget{produit-scalaireproduit-scalaire}{%
\subsubsection{{[}{[}produit scalaire\textbar Produit
scalaire{]}{]}}\label{produit-scalaireproduit-scalaire}}

\[
\langle x, y \rangle = \frac{1}{T} \int^{-T/2}_{T/2} x y^\ast
\]

\hypertarget{densituxe9-spectrale-duxe9nergiedensituxe9-spectrale-de-puissance}{%
\subsubsection{{[}{[}densité spectrale
d\textquotesingle énergie\textbar densité spectrale de
puissance{]}{]}}\label{densituxe9-spectrale-duxe9nergiedensituxe9-spectrale-de-puissance}}

Soit
\(x(t) = \displaystyle\sum_{k\in \mathbb{Z}} c_k \exp(j2\pi k f_0 t)\)
Alors: \[
s_x(f) = \sum_{k\in \mathbb{Z}} |c_k|^2 \delta(f - kf_0)
\]

\hypertarget{signal-duxe9terministe-non-puxe9riodique-uxe0-puissance-finie}{%
\subsection{signal déterministe non périodique à puissance
finie}\label{signal-duxe9terministe-non-puxe9riodique-uxe0-puissance-finie}}

\hypertarget{duxe9finition-2}{%
\subsubsection{Définition}\label{duxe9finition-2}}

Un {[}{[}signal{]}{]} {[}{[}signal
déterministe\textbar déterministe{]}{]} non-{[}{[}périodique{]}{]} à
{[}{[}énergie d\textquotesingle un signal\textbar puissance{]}{]}
{[}{[}fini\textbar finie{]}{]}

\hypertarget{puissance-1}{%
\subsubsection{Puissance}\label{puissance-1}}

\[
P(x) = \lim_{T\to \infty} \frac{1}{T} \int^{-T/2}_{T/2} |x|^2
\]

\hypertarget{produit-scalaireproduit-scalaire-1}{%
\subsubsection{{[}{[}produit scalaire\textbar Produit
scalaire{]}{]}}\label{produit-scalaireproduit-scalaire-1}}

\[
\langle x, y\rangle = \lim_{T\to \infty} \frac{1}{T} \int^{-T/2}_{T/2} xy^\ast
\]

\hypertarget{densituxe9-spectrale-duxe9nergiedensituxe9-spectrale-de-puissance-1}{%
\subsubsection{{[}{[}densité spectrale
d\textquotesingle énergie\textbar densité spectrale de
puissance{]}{]}}\label{densituxe9-spectrale-duxe9nergiedensituxe9-spectrale-de-puissance-1}}

\[
s_x(f) = \lim_{T\to \infty} \frac{1}{T} \left| \int^{-T/2}_{T/2} x(t) e^{-j2\pi ft} \mathrm{d}t \right|^2
\]

\hypertarget{signal-aluxe9atoire-stationnaire}{%
\subsection{signal aléatoire
stationnaire}\label{signal-aluxe9atoire-stationnaire}}

\hypertarget{duxe9finition-3}{%
\subsubsection{Définition}\label{duxe9finition-3}}

Un {[}{[}signal{]}{]} non-{[}{[}signal
déterministe\textbar déterministe{]}{]}
{[}{[}stationnarité\textbar stationnaire{]}{]}. Il est défini par une
{[}{[}variable aléatoire{]}{]} au lieu d\textquotesingle une fonction du
temps.

\begin{itemize}
\tightlist
\item
  La {[}{[}espérance\textbar moyenne{]}{]} ne dépend pas de \(t\)
\item
  Le {[}{[}moment d\textquotesingle ordre 2{]}{]}
  \(E(t\mapsto x(t) x^\ast(t-\tau))\) ne dépend pas de \(t\)
\end{itemize}

\hypertarget{produit-scalaire-1}{%
\subsubsection{{[}{[}produit scalaire{]}{]}}\label{produit-scalaire-1}}

\[
\langle x, y \rangle = E(xy^\ast)
\]

\hypertarget{uxe9nergie-dun-signalpuissance-moyenne}{%
\subsubsection{{[}{[}énergie d\textquotesingle un
signal\textbar puissance{]}{]}
moyenne}\label{uxe9nergie-dun-signalpuissance-moyenne}}

\[
P(x) = E(|x|^2)
\]

\hypertarget{densituxe9-spectrale-duxe9nergiedensituxe9-spectrale-de-puissance-2}{%
\subsubsection{{[}{[}densité spectrale
d\textquotesingle énergie\textbar densité spectrale de
puissance{]}{]}}\label{densituxe9-spectrale-duxe9nergiedensituxe9-spectrale-de-puissance-2}}

\textbf{Si \(X = \mathcal{F}[x]\) existe}:

\[
s_x(f) = \lim_{T\to \infty} \frac{1}{T} E\left( \left| \int^{-T/2}_{T/2} X(f) e^{-j2\pi ft} \mathrm{d}t \right|^2 \right)
\]

\hypertarget{espuxe9rance-dun-signal-filtrefiltruxe9}{%
\subsubsection{Espérance d\textquotesingle un signal
{[}{[}filtre\textbar filtré{]}{]}}\label{espuxe9rance-dun-signal-filtrefiltruxe9}}

\[
E(H \cdot X) = H(0) \cdot E(X)
\]

\hypertarget{densituxe9-spectrale-duxe9nergie-1}{%
\subsection{densité spectrale
d\textquotesingle énergie}\label{densituxe9-spectrale-duxe9nergie-1}}

Pour n\textquotesingle importe quel {[}{[}signal{]}{]},
c\textquotesingle est la {[}{[}transformée de Fourrier{]}{]} de
l\textquotesingle{[}{[}autocorrélation{]}{]}: \[
s_x = \mathcal{F}[R_x]
\]

Il y a aussi une caractérisation qui diffère selon la {[}{[}classes de
signaux\textbar classe du signal{]}{]}.

\hypertarget{propriuxe9tuxe9s-4}{%
\subsubsection{Propriétés}\label{propriuxe9tuxe9s-4}}

\begin{itemize}
\tightlist
\item
  \textbf{positive} \(s_x \geq 0\)
\item
  \textbf{réelle} \(s_x \in \mathbb{R}^\mathbb{R}\)
\item
  \textbf{parité} si \(x\) réel, \(s_x\) {[}{[}parité
  d\textquotesingle une fonction\textbar paire{]}{]}
\end{itemize}

\hypertarget{lien-avec-luxe9nergie-dun-signal}{%
\subsubsection{Lien avec l\textquotesingle{[}{[}énergie
d\textquotesingle un
signal{]}{]}}\label{lien-avec-luxe9nergie-dun-signal}}

\[
\int_\mathbb{R} s_x = E(x)
\]

\hypertarget{du-signal-filtrefiltruxe9}{%
\subsubsection{Du signal
{[}{[}filtre\textbar filtré{]}{]}}\label{du-signal-filtrefiltruxe9}}

Voir les {[}{[}relations de Wiener-Lee{]}{]}

\hypertarget{lien-avec-les-probabilituxe9s}{%
\subsubsection{Lien avec les
probabilités}\label{lien-avec-les-probabilituxe9s}}

S\textquotesingle apparente à la {[}{[}densité de probabilité{]}{]}:
c\textquotesingle est la répartition de l\textquotesingle énergie en
fonction des fréquences

\hypertarget{intercorruxe9lation}{%
\subsection{intercorrélation}\label{intercorruxe9lation}}

Pour deux {[}{[}signal\textbar signaux{]}{]} \(x\) et \(y\) de même
{[}{[}classes de signaux\textbar classe{]}{]}

\[
R_{xy}(\tau) = \left\langle x, y(\cdot-\tau)\right\rangle\] Avec
\(\langle \cdot, \cdot\cdot \rangle\) dépendant de la classe des signaux

\hypertarget{du-signal-filtrefiltruxe9-1}{%
\subsubsection{Du signal
{[}{[}filtre\textbar filtré{]}{]}}\label{du-signal-filtrefiltruxe9-1}}

Voir les {[}{[}relations de Wiener-Lee{]}{]}

\hypertarget{autocorruxe9lation}{%
\subsection{autocorrélation}\label{autocorruxe9lation}}

Cas particulier de l\textquotesingle{[}{[}intercorrélation{]}{]}

\[
R_x(\tau) = R_{xx}(\tau)
\]

\hypertarget{du-signal-filtrefiltruxe9-2}{%
\subsubsection{Du signal
{[}{[}filtre\textbar filtré{]}{]}}\label{du-signal-filtrefiltruxe9-2}}

Voir les {[}{[}relations de Wiener-Lee{]}{]}

\hypertarget{symuxe9trie-hermitienne}{%
\subsubsection{Symétrie Hermitienne}\label{symuxe9trie-hermitienne}}

\[
R_x^\ast(-\tau) = R_x(\tau)
\]

\hypertarget{valeur-maximale}{%
\subsubsection{Valeur maximale}\label{valeur-maximale}}

\[
|R_x(\tau)| \leq R_x(0)
\]

\hypertarget{distance-entre-xt-et-xt-tau}{%
\subsubsection{\texorpdfstring{Distance entre \(x(t)\) et
\(x(t-\tau)\)}{Distance entre x(t) et x(t-\textbackslash tau)}}\label{distance-entre-xt-et-xt-tau}}

\[
d(x(t), x(t-\tau)) = \sqrt{ 2\Big[R_x(0) - R_x(\tau)\Big] }
\]

\hypertarget{duxe9composition-de-lebesgue}{%
\subsubsection{Décomposition de
Lebesgue}\label{duxe9composition-de-lebesgue}}

On peut quasiment tout le temps décomposer
l\textquotesingle autocorrélation en une somme d\textquotesingle une
fonction \(\xrightarrow[\tau \to \infty]{} 0\) et d\textquotesingle une
somme de fonctions {[}{[}périodique\textbar périodiques{]}{]}

\hypertarget{signal-bruit}{%
\subsection{signal bruit}\label{signal-bruit}}

\hypertarget{bruit-blanc}{%
\subsubsection{Bruit blanc}\label{bruit-blanc}}

{[}{[}signal aléatoire stationnaire{]}{]} tel que \[
\begin{cases}
R_x &= \frac{N_0}{2} \delta \\
s_x &= \frac{N_0}{2}
\end{cases}
\]

\hypertarget{bruit-gaussien}{%
\subsubsection{Bruit gaussien}\label{bruit-gaussien}}

\hypertarget{ruxe9alisabilituxe9-dun-filtre}{%
\subsection{réalisabilité d\textquotesingle un
filtre}\label{ruxe9alisabilituxe9-dun-filtre}}

\hypertarget{duxe9finition-4}{%
\subsubsection{Définition}\label{duxe9finition-4}}

\hypertarget{domaine-temporel}{%
\paragraph{Domaine temporel}\label{domaine-temporel}}

Un filtre de {[}{[}réponse impulsionnelle{]}{]} \(h\) est dit
\emph{réalisable} si et seulement si:

\begin{enumerate}
\def\labelenumi{\arabic{enumi}.}
\tightlist
\item
  \(h\) est réelle
\item
  \(h \in L^1\) (stabilité)
\item
  \(h\) est {[}{[}réponse impulsionnelle causale\textbar causale{]}{]}
\end{enumerate}

\hypertarget{domaine-fruxe9quentiel}{%
\paragraph{Domaine fréquentiel}\label{domaine-fruxe9quentiel}}

\begin{enumerate}
\def\labelenumi{\arabic{enumi}.}
\tightlist
\item
  \(H\) vérifie la {[}{[}symétrie Hermitienne{]}{]}
\item
  ne peut se traduire
\item
  \(H = -jH \ast \frac{1}{\pi \operatorname{id}}\)
\end{enumerate}

\hypertarget{relation-de-filtrage-linuxe9aire}{%
\subsection{relation de filtrage
linéaire}\label{relation-de-filtrage-linuxe9aire}}

\#signal

\hypertarget{signal-duxe9terministe}{%
\paragraph{{[}{[}signal
déterministe{]}{]}}\label{signal-duxe9terministe}}

\[
Y = X \cdot H
\]

\hypertarget{signal-aluxe9atoire-stationnaire-isomuxe9trie-fondamentale}{%
\paragraph{{[}{[}signal aléatoire stationnaire{]}{]}: isométrie
fondamentale}\label{signal-aluxe9atoire-stationnaire-isomuxe9trie-fondamentale}}

TODO diapo 36 c\textquotesingle est quoi
\(\stackrel{I}{\leftrightarrow}\) ?

\hypertarget{relations-de-wiener-lee}{%
\subsection{relations de Wiener-Lee}\label{relations-de-wiener-lee}}

\#signal

On pose \(x\) un {[}{[}signal{]}{]} et \(h\) la {[}{[}réponse
impulsionnelle{]}{]} d\textquotesingle un {[}{[}filtre{]}{]}.

\hypertarget{sur-la-densituxe9-spectrale-duxe9nergie}{%
\subsubsection{Sur la {[}{[}densité spectrale
d\textquotesingle énergie{]}{]}}\label{sur-la-densituxe9-spectrale-duxe9nergie}}

\[
s_{x * h} = |H|^2 \cdot s_x
\]

\hypertarget{sur-lintercorruxe9lation}{%
\subsubsection{Sur
l\textquotesingle{[}{[}intercorrélation{]}{]}}\label{sur-lintercorruxe9lation}}

\[
R_{x\ast h,\ x} = R_x \ast h
\]

\hypertarget{sur-lautocorruxe9lation}{%
\subsubsection{Sur
l\textquotesingle{[}{[}autocorrélation{]}{]}}\label{sur-lautocorruxe9lation}}

\[
R_{x\ast h}(\tau) = R_x(\tau) \ast h(\tau) \ast h^\ast(-\tau)
\]

\hypertarget{quantificateur-de-signal}{%
\subsection{quantificateur de signal}\label{quantificateur-de-signal}}

Une {[}{[}transformation non-linéaire d\textquotesingle un signal{]}{]}
avec \[
y(t) = i(t) \Delta q_{i(t)}
\]

Avec:

\begin{itemize}
\tightlist
\item
  \(i(t)\) l\textquotesingle indice de l\textquotesingle échantillon le
  plus proche temporellement de \(t\)
\item
  \(\Delta q_{i(t)}\) tel que \[
  x(t) \in x_{i(t)} + \frac{1}{2} \left[ -\Delta q_{i(t)}, +\Delta q_{i(t)} \right]
  \]
\end{itemize}

\hypertarget{quantification-uniforme}{%
\subsubsection{Quantification uniforme}\label{quantification-uniforme}}

Quand \(\Delta q_{i(t)}\) ne dépend pas de \(t\) (donc ni de \(i\)):

\[
\Delta q_{i(t)} = \Delta q =  \frac{2 \max x}{2^\text{bits}}
\]

\hypertarget{erreur-de-quantification}{%
\subsubsection{Erreur de
quantification}\label{erreur-de-quantification}}

On suppose que l\textquotesingle erreur \(\varepsilon\) suit
\(\mathcal{U}([ -\Delta q, +\Delta q ]/2)\).

\hypertarget{snr}{%
\paragraph{SNR}\label{snr}}

\begin{align*}
\operatorname{SNR}_{\text{dB}} &= 10 \log \frac{\sigma_x^2}{\sigma_\varepsilon^2} \\
&= 10 \log \frac{(\max x)^2 /2}{(\Delta q)^2 / 12} \\
&= 10 \log \left(  \frac{(\max x)^2}{2} \cdot \frac{12 \cdot 2^{\text{2bits}}}{4 (\max x)^2} \right) \\
&= 10 \log \left( \frac{12}{8}  2^{2\text{bits}} \right ) \\
&= 10 \log \frac{8}{12} + 10 \log 2^{2 \text{bits}} \\
&\stackrel{\text{approx?}}{=} 1.76 + 6\text{bits}
\end{align*}

\hypertarget{transformation-non-linuxe9aire-dun-signal}{%
\subsection{transformation non-linéaire d\textquotesingle un
signal}\label{transformation-non-linuxe9aire-dun-signal}}

Sorte de "filtre" mais sans {[}{[}relation de filtrage linéaire{]}{]},
la sortie l\textquotesingle image d\textquotesingle une
{[}{[}fonction{]}{]} quelconque de l\textquotesingle entrée:

\[
y = g(x)
\]

\hypertarget{exemples}{%
\subsubsection{Exemples}\label{exemples}}

\begin{itemize}
\tightlist
\item
  {[}{[}quadrateur{]}{]}
\item
  {[}{[}quantificateur de signal{]}{]}
\end{itemize}

\hypertarget{thuxe9oruxe8me-de-price}{%
\subsection{théorème de Price}\label{thuxe9oruxe8me-de-price}}

\#signal

\hypertarget{uxe9noncuxe9}{%
\subsubsection{Énoncé}\label{uxe9noncuxe9}}

Soit \((X_1, X_2) \sim \mathcal{N}_2(m, \sigma^2)\) avec \(m \neq 0\) et
\(g: \mathbb{R} \to \mathbb{R}\)

\[
\frac{\partial E(g(X_1) g(X_2))}{\partial E(X_1 X_2)} = E\left( \frac{\partial g(X_1)}{\partial X_1} \cdot \frac{\partial g(X_2)}{X_2}  \right)
\]

\hypertarget{application-au-quadrateur}{%
\subsubsection{Application au
{[}{[}quadrateur{]}{]}}\label{application-au-quadrateur}}

On a \(y = x^2\).

On veut une expression de \(R_y\) en fonction de \(R_x\).

\begin{align*}
\frac{\partial R_y}{\partial R_x}(\tau)  &= \frac{\partial \langle y, y(\cdot -\tau) \rangle }{\partial \langle x, x(\cdot -\tau) \rangle} \\
&= \frac{\partial E(x^2(t) x^2(t-\tau))}{\partial E(x(t) x(t-\tau))} \\
&= \frac{\partial E(g(x_1(t)) g(x_2(t)) )}{\partial E(x_1(t) x_2(t))} \quad \text{avec}\ x_1 = x,\ x_2 = x(\cdot - \tau) \text{ et } g = \operatorname{id}^2 \\
&= E\left( \frac{\partial x^2}{\partial x} \cdot \frac{\partial x(\cdot - \tau)^2}{\partial x(\cdot - \tau)} \right) \quad \text{d'après le théorème de Price} \\
&= E(4x(t)x(t-\tau)) \\
&= 4 E(x(t) x(t-\tau)) \\
&= 4 \langle x, x(\cdot - \tau) \rangle \\
&= 4 R_x(\tau)
\end{align*}

On intègre:

\begin{align*}
R_y(\tau) &= 2R_x(\tau)^2 + \text{const}
\end{align*}

Pour trouver \(\text{const}\), on évalue en \(\tau = 0\):

\begin{align*}
R_y(0) &= \langle y, y \rangle \\
&= E(yy^\ast) \\
&= E(|y|^2) \\
&= E(|x^2|^2) \\
&= E(x^4) \\
&= (3 \times 1) \sigma^4 \quad \text{moments d'une loi Gaussienne centrée} \\
&= 4\sigma^4 \\
\text{or} \quad R_y(0) &= 2R_x^2(0) + \text{const} \\
\text{donc} \quad \text{const} &= 4\sigma^4 - 2(R_x(0))^2 \\
&= 4\sigma^4 - 2E(|x|^2) \\
&= 4\sigma^4 - \sigma^2 \quad \text{si $|x| = x$ i.e. $x\geq 0$}
\end{align*}

On peut aussi s\textquotesingle en tenir à \(\text{const} = R^2_x(0)\)

\end{document}
